% Options for packages loaded elsewhere
% Options for packages loaded elsewhere
\PassOptionsToPackage{unicode}{hyperref}
\PassOptionsToPackage{hyphens}{url}
%
\documentclass[
]{book}
\usepackage{xcolor}
\usepackage{amsmath,amssymb}
\setcounter{secnumdepth}{-\maxdimen} % remove section numbering
\usepackage{iftex}
\ifPDFTeX
  \usepackage[T1]{fontenc}
  \usepackage[utf8]{inputenc}
  \usepackage{textcomp} % provide euro and other symbols
\else % if luatex or xetex
  \usepackage{unicode-math} % this also loads fontspec
  \defaultfontfeatures{Scale=MatchLowercase}
  \defaultfontfeatures[\rmfamily]{Ligatures=TeX,Scale=1}
\fi
\usepackage{lmodern}
\ifPDFTeX\else
  % xetex/luatex font selection
\fi
% Use upquote if available, for straight quotes in verbatim environments
\IfFileExists{upquote.sty}{\usepackage{upquote}}{}
\IfFileExists{microtype.sty}{% use microtype if available
  \usepackage[]{microtype}
  \UseMicrotypeSet[protrusion]{basicmath} % disable protrusion for tt fonts
}{}
\makeatletter
\@ifundefined{KOMAClassName}{% if non-KOMA class
  \IfFileExists{parskip.sty}{%
    \usepackage{parskip}
  }{% else
    \setlength{\parindent}{0pt}
    \setlength{\parskip}{6pt plus 2pt minus 1pt}}
}{% if KOMA class
  \KOMAoptions{parskip=half}}
\makeatother
% Make \paragraph and \subparagraph free-standing
\makeatletter
\ifx\paragraph\undefined\else
  \let\oldparagraph\paragraph
  \renewcommand{\paragraph}{
    \@ifstar
      \xxxParagraphStar
      \xxxParagraphNoStar
  }
  \newcommand{\xxxParagraphStar}[1]{\oldparagraph*{#1}\mbox{}}
  \newcommand{\xxxParagraphNoStar}[1]{\oldparagraph{#1}\mbox{}}
\fi
\ifx\subparagraph\undefined\else
  \let\oldsubparagraph\subparagraph
  \renewcommand{\subparagraph}{
    \@ifstar
      \xxxSubParagraphStar
      \xxxSubParagraphNoStar
  }
  \newcommand{\xxxSubParagraphStar}[1]{\oldsubparagraph*{#1}\mbox{}}
  \newcommand{\xxxSubParagraphNoStar}[1]{\oldsubparagraph{#1}\mbox{}}
\fi
\makeatother


\usepackage{longtable,booktabs,array}
\usepackage{calc} % for calculating minipage widths
% Correct order of tables after \paragraph or \subparagraph
\usepackage{etoolbox}
\makeatletter
\patchcmd\longtable{\par}{\if@noskipsec\mbox{}\fi\par}{}{}
\makeatother
% Allow footnotes in longtable head/foot
\IfFileExists{footnotehyper.sty}{\usepackage{footnotehyper}}{\usepackage{footnote}}
\makesavenoteenv{longtable}
\usepackage{graphicx}
\makeatletter
\newsavebox\pandoc@box
\newcommand*\pandocbounded[1]{% scales image to fit in text height/width
  \sbox\pandoc@box{#1}%
  \Gscale@div\@tempa{\textheight}{\dimexpr\ht\pandoc@box+\dp\pandoc@box\relax}%
  \Gscale@div\@tempb{\linewidth}{\wd\pandoc@box}%
  \ifdim\@tempb\p@<\@tempa\p@\let\@tempa\@tempb\fi% select the smaller of both
  \ifdim\@tempa\p@<\p@\scalebox{\@tempa}{\usebox\pandoc@box}%
  \else\usebox{\pandoc@box}%
  \fi%
}
% Set default figure placement to htbp
\def\fps@figure{htbp}
\makeatother





\setlength{\emergencystretch}{3em} % prevent overfull lines

\providecommand{\tightlist}{%
  \setlength{\itemsep}{0pt}\setlength{\parskip}{0pt}}



 


% ---- Basic math setup ----
\usepackage{fontspec} % keep for Unicode, harmless even without custom fonts
% \setmainfont{TeX Gyre Pagella}      % comment out
% \setmathfont{TeX Gyre Pagella Math} % comment out

% ---- Disable emojis ----
\newcommand{\emoji}[1]{}

% ---- Math packages ----
\usepackage{amsmath,amssymb}
\usepackage{tikz}
\usetikzlibrary{arrows.meta,shapes,positioning}
\makeatletter
\@ifpackageloaded{caption}{}{\usepackage{caption}}
\AtBeginDocument{%
\ifdefined\contentsname
  \renewcommand*\contentsname{Table of contents}
\else
  \newcommand\contentsname{Table of contents}
\fi
\ifdefined\listfigurename
  \renewcommand*\listfigurename{List of Figures}
\else
  \newcommand\listfigurename{List of Figures}
\fi
\ifdefined\listtablename
  \renewcommand*\listtablename{List of Tables}
\else
  \newcommand\listtablename{List of Tables}
\fi
\ifdefined\figurename
  \renewcommand*\figurename{Figure}
\else
  \newcommand\figurename{Figure}
\fi
\ifdefined\tablename
  \renewcommand*\tablename{Table}
\else
  \newcommand\tablename{Table}
\fi
}
\@ifpackageloaded{float}{}{\usepackage{float}}
\floatstyle{ruled}
\@ifundefined{c@chapter}{\newfloat{codelisting}{h}{lop}}{\newfloat{codelisting}{h}{lop}[chapter]}
\floatname{codelisting}{Listing}
\newcommand*\listoflistings{\listof{codelisting}{List of Listings}}
\makeatother
\makeatletter
\makeatother
\makeatletter
\@ifpackageloaded{caption}{}{\usepackage{caption}}
\@ifpackageloaded{subcaption}{}{\usepackage{subcaption}}
\makeatother
\usepackage{bookmark}
\IfFileExists{xurl.sty}{\usepackage{xurl}}{} % add URL line breaks if available
\urlstyle{same}
\hypersetup{
  hidelinks,
  pdfcreator={LaTeX via pandoc}}


\author{}
\date{}
\begin{document}
\frontmatter


\mainmatter
\chapter{Level 3: PDEs, their Physical Meaning, and Boundary
Conditions}\label{level-3-pdes-their-physical-meaning-and-boundary-conditions}

\textbf{Pavni:} Acharya, PDEs still feel mysterious. What do they
\emph{really} mean?

\textbf{Acharya:} Good question. PDEs describe how a quantity changes
with respect to \textbf{both time and space}. Think of:\\
- Heat spreading in a rod\\
- Waves traveling along a string\\
- Fluid flowing through a pipe

All of these involve rates of change in multiple directions, and that's
why PDEs come into play.

\textbf{Pavni:} So PDEs are the language of physics in extended domains?

\textbf{Acharya:} Exactly. But to make their solutions unique and
physically meaningful, we need \textbf{boundary conditions}. Let's
explore them one by one.

\begin{center}\rule{0.5\linewidth}{0.5pt}\end{center}

\section{Dirichlet Condition}\label{dirichlet-condition}

\textbf{Acharya:} Dirichlet means fixing the value of the solution at
the boundary.

\textbf{Pavni:} Like holding both ends of a rod at 100 °C?

\textbf{Acharya:} Precisely. It represents physical situations where the
boundary is controlled by an external source---like contact with a
thermostat.

\begin{center}\rule{0.5\linewidth}{0.5pt}\end{center}

\section{Neumann Condition}\label{neumann-condition}

\textbf{Acharya:} Neumann means fixing the derivative, often
representing \textbf{flux}.

\textbf{Pavni:} So in the rod, saying no heat flows out means the
temperature gradient at the end is zero?

\textbf{Acharya:} Exactly. That's an insulated boundary.

\begin{center}\rule{0.5\linewidth}{0.5pt}\end{center}

\section{Robin (Mixed) Condition}\label{robin-mixed-condition}

\textbf{Acharya:} Robin mixes the two:\\
\[
a u + b \frac{\partial u}{\partial n} = c.
\]

\textbf{Pavni:} Is that like when heat escapes to the air?

\textbf{Acharya:} Yes---convective cooling. The flux depends on both the
temperature at the boundary and the environment.

\begin{center}\rule{0.5\linewidth}{0.5pt}\end{center}

💡 Quick Recap

\begin{itemize}
\tightlist
\item
  \textbf{Dirichlet} → Value fixed (e.g., temperature = 100 °C).\\
\item
  \textbf{Neumann} → Flux fixed (e.g., insulated boundary).\\
\item
  \textbf{Robin} → Combination (e.g., convective heat loss).
\end{itemize}

\begin{center}\rule{0.5\linewidth}{0.5pt}\end{center}

\subsection{📝 Mini-Quiz}\label{mini-quiz}

\begin{enumerate}
\def\labelenumi{\arabic{enumi}.}
\tightlist
\item
  A vibrating string held fixed at both ends uses which boundary
  condition?\\

  Answer

  \textbf{Dirichlet.} The displacement of the string is zero at both
  ends.
\item
  If a wall is perfectly insulated, what type of boundary condition
  applies to temperature?\\

  Answer

  \textbf{Neumann.} The derivative (temperature gradient) is zero,
  meaning no heat flux.
\item
  Which boundary condition models cooling of hot coffee in a room?\\

  Answer

  \textbf{Robin.} Heat loss depends on both the coffee's surface
  temperature and the room temperature (convection).
\end{enumerate}

\begin{center}\rule{0.5\linewidth}{0.5pt}\end{center}

\textbf{Pavni:} Now I see it! PDEs tell the story inside the domain, and
boundary conditions set the rules at the edges.

\textbf{Acharya:} Well said. Together, they form the complete model of a
physical system.

\begin{center}\rule{0.5\linewidth}{0.5pt}\end{center}


\backmatter


\end{document}
