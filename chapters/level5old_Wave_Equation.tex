% Options for packages loaded elsewhere
% Options for packages loaded elsewhere
\PassOptionsToPackage{unicode}{hyperref}
\PassOptionsToPackage{hyphens}{url}
%
\documentclass[
]{book}
\usepackage{xcolor}
\usepackage{amsmath,amssymb}
\setcounter{secnumdepth}{-\maxdimen} % remove section numbering
\usepackage{iftex}
\ifPDFTeX
  \usepackage[T1]{fontenc}
  \usepackage[utf8]{inputenc}
  \usepackage{textcomp} % provide euro and other symbols
\else % if luatex or xetex
  \usepackage{unicode-math} % this also loads fontspec
  \defaultfontfeatures{Scale=MatchLowercase}
  \defaultfontfeatures[\rmfamily]{Ligatures=TeX,Scale=1}
\fi
\usepackage{lmodern}
\ifPDFTeX\else
  % xetex/luatex font selection
\fi
% Use upquote if available, for straight quotes in verbatim environments
\IfFileExists{upquote.sty}{\usepackage{upquote}}{}
\IfFileExists{microtype.sty}{% use microtype if available
  \usepackage[]{microtype}
  \UseMicrotypeSet[protrusion]{basicmath} % disable protrusion for tt fonts
}{}
\makeatletter
\@ifundefined{KOMAClassName}{% if non-KOMA class
  \IfFileExists{parskip.sty}{%
    \usepackage{parskip}
  }{% else
    \setlength{\parindent}{0pt}
    \setlength{\parskip}{6pt plus 2pt minus 1pt}}
}{% if KOMA class
  \KOMAoptions{parskip=half}}
\makeatother
% Make \paragraph and \subparagraph free-standing
\makeatletter
\ifx\paragraph\undefined\else
  \let\oldparagraph\paragraph
  \renewcommand{\paragraph}{
    \@ifstar
      \xxxParagraphStar
      \xxxParagraphNoStar
  }
  \newcommand{\xxxParagraphStar}[1]{\oldparagraph*{#1}\mbox{}}
  \newcommand{\xxxParagraphNoStar}[1]{\oldparagraph{#1}\mbox{}}
\fi
\ifx\subparagraph\undefined\else
  \let\oldsubparagraph\subparagraph
  \renewcommand{\subparagraph}{
    \@ifstar
      \xxxSubParagraphStar
      \xxxSubParagraphNoStar
  }
  \newcommand{\xxxSubParagraphStar}[1]{\oldsubparagraph*{#1}\mbox{}}
  \newcommand{\xxxSubParagraphNoStar}[1]{\oldsubparagraph{#1}\mbox{}}
\fi
\makeatother

\usepackage{color}
\usepackage{fancyvrb}
\newcommand{\VerbBar}{|}
\newcommand{\VERB}{\Verb[commandchars=\\\{\}]}
\DefineVerbatimEnvironment{Highlighting}{Verbatim}{commandchars=\\\{\}}
% Add ',fontsize=\small' for more characters per line
\usepackage{framed}
\definecolor{shadecolor}{RGB}{241,243,245}
\newenvironment{Shaded}{\begin{snugshade}}{\end{snugshade}}
\newcommand{\AlertTok}[1]{\textcolor[rgb]{0.68,0.00,0.00}{#1}}
\newcommand{\AnnotationTok}[1]{\textcolor[rgb]{0.37,0.37,0.37}{#1}}
\newcommand{\AttributeTok}[1]{\textcolor[rgb]{0.40,0.45,0.13}{#1}}
\newcommand{\BaseNTok}[1]{\textcolor[rgb]{0.68,0.00,0.00}{#1}}
\newcommand{\BuiltInTok}[1]{\textcolor[rgb]{0.00,0.23,0.31}{#1}}
\newcommand{\CharTok}[1]{\textcolor[rgb]{0.13,0.47,0.30}{#1}}
\newcommand{\CommentTok}[1]{\textcolor[rgb]{0.37,0.37,0.37}{#1}}
\newcommand{\CommentVarTok}[1]{\textcolor[rgb]{0.37,0.37,0.37}{\textit{#1}}}
\newcommand{\ConstantTok}[1]{\textcolor[rgb]{0.56,0.35,0.01}{#1}}
\newcommand{\ControlFlowTok}[1]{\textcolor[rgb]{0.00,0.23,0.31}{\textbf{#1}}}
\newcommand{\DataTypeTok}[1]{\textcolor[rgb]{0.68,0.00,0.00}{#1}}
\newcommand{\DecValTok}[1]{\textcolor[rgb]{0.68,0.00,0.00}{#1}}
\newcommand{\DocumentationTok}[1]{\textcolor[rgb]{0.37,0.37,0.37}{\textit{#1}}}
\newcommand{\ErrorTok}[1]{\textcolor[rgb]{0.68,0.00,0.00}{#1}}
\newcommand{\ExtensionTok}[1]{\textcolor[rgb]{0.00,0.23,0.31}{#1}}
\newcommand{\FloatTok}[1]{\textcolor[rgb]{0.68,0.00,0.00}{#1}}
\newcommand{\FunctionTok}[1]{\textcolor[rgb]{0.28,0.35,0.67}{#1}}
\newcommand{\ImportTok}[1]{\textcolor[rgb]{0.00,0.46,0.62}{#1}}
\newcommand{\InformationTok}[1]{\textcolor[rgb]{0.37,0.37,0.37}{#1}}
\newcommand{\KeywordTok}[1]{\textcolor[rgb]{0.00,0.23,0.31}{\textbf{#1}}}
\newcommand{\NormalTok}[1]{\textcolor[rgb]{0.00,0.23,0.31}{#1}}
\newcommand{\OperatorTok}[1]{\textcolor[rgb]{0.37,0.37,0.37}{#1}}
\newcommand{\OtherTok}[1]{\textcolor[rgb]{0.00,0.23,0.31}{#1}}
\newcommand{\PreprocessorTok}[1]{\textcolor[rgb]{0.68,0.00,0.00}{#1}}
\newcommand{\RegionMarkerTok}[1]{\textcolor[rgb]{0.00,0.23,0.31}{#1}}
\newcommand{\SpecialCharTok}[1]{\textcolor[rgb]{0.37,0.37,0.37}{#1}}
\newcommand{\SpecialStringTok}[1]{\textcolor[rgb]{0.13,0.47,0.30}{#1}}
\newcommand{\StringTok}[1]{\textcolor[rgb]{0.13,0.47,0.30}{#1}}
\newcommand{\VariableTok}[1]{\textcolor[rgb]{0.07,0.07,0.07}{#1}}
\newcommand{\VerbatimStringTok}[1]{\textcolor[rgb]{0.13,0.47,0.30}{#1}}
\newcommand{\WarningTok}[1]{\textcolor[rgb]{0.37,0.37,0.37}{\textit{#1}}}

\usepackage{longtable,booktabs,array}
\usepackage{calc} % for calculating minipage widths
% Correct order of tables after \paragraph or \subparagraph
\usepackage{etoolbox}
\makeatletter
\patchcmd\longtable{\par}{\if@noskipsec\mbox{}\fi\par}{}{}
\makeatother
% Allow footnotes in longtable head/foot
\IfFileExists{footnotehyper.sty}{\usepackage{footnotehyper}}{\usepackage{footnote}}
\makesavenoteenv{longtable}
\usepackage{graphicx}
\makeatletter
\newsavebox\pandoc@box
\newcommand*\pandocbounded[1]{% scales image to fit in text height/width
  \sbox\pandoc@box{#1}%
  \Gscale@div\@tempa{\textheight}{\dimexpr\ht\pandoc@box+\dp\pandoc@box\relax}%
  \Gscale@div\@tempb{\linewidth}{\wd\pandoc@box}%
  \ifdim\@tempb\p@<\@tempa\p@\let\@tempa\@tempb\fi% select the smaller of both
  \ifdim\@tempa\p@<\p@\scalebox{\@tempa}{\usebox\pandoc@box}%
  \else\usebox{\pandoc@box}%
  \fi%
}
% Set default figure placement to htbp
\def\fps@figure{htbp}
\makeatother





\setlength{\emergencystretch}{3em} % prevent overfull lines

\providecommand{\tightlist}{%
  \setlength{\itemsep}{0pt}\setlength{\parskip}{0pt}}



 


% ---- Basic math setup ----
\usepackage{fontspec} % keep for Unicode, harmless even without custom fonts
% \setmainfont{TeX Gyre Pagella}      % comment out
% \setmathfont{TeX Gyre Pagella Math} % comment out

% ---- Disable emojis ----
\newcommand{\emoji}[1]{}

% ---- Math packages ----
\usepackage{amsmath,amssymb}
\usepackage{tikz}
\usetikzlibrary{arrows.meta,shapes,positioning}
\makeatletter
\@ifpackageloaded{caption}{}{\usepackage{caption}}
\AtBeginDocument{%
\ifdefined\contentsname
  \renewcommand*\contentsname{Table of contents}
\else
  \newcommand\contentsname{Table of contents}
\fi
\ifdefined\listfigurename
  \renewcommand*\listfigurename{List of Figures}
\else
  \newcommand\listfigurename{List of Figures}
\fi
\ifdefined\listtablename
  \renewcommand*\listtablename{List of Tables}
\else
  \newcommand\listtablename{List of Tables}
\fi
\ifdefined\figurename
  \renewcommand*\figurename{Figure}
\else
  \newcommand\figurename{Figure}
\fi
\ifdefined\tablename
  \renewcommand*\tablename{Table}
\else
  \newcommand\tablename{Table}
\fi
}
\@ifpackageloaded{float}{}{\usepackage{float}}
\floatstyle{ruled}
\@ifundefined{c@chapter}{\newfloat{codelisting}{h}{lop}}{\newfloat{codelisting}{h}{lop}[chapter]}
\floatname{codelisting}{Listing}
\newcommand*\listoflistings{\listof{codelisting}{List of Listings}}
\makeatother
\makeatletter
\makeatother
\makeatletter
\@ifpackageloaded{caption}{}{\usepackage{caption}}
\@ifpackageloaded{subcaption}{}{\usepackage{subcaption}}
\makeatother
\usepackage{bookmark}
\IfFileExists{xurl.sty}{\usepackage{xurl}}{} % add URL line breaks if available
\urlstyle{same}
\hypersetup{
  pdftitle={Level 4: Hyperbolic PDEs: Wave Equation},
  hidelinks,
  pdfcreator={LaTeX via pandoc}}


\title{Level 4: Hyperbolic PDEs: Wave Equation}
\author{}
\date{}
\begin{document}
\frontmatter
\maketitle


\mainmatter
\textbf{Pavni:} We've studied the heat equation as an example of a
parabolic PDE. What about hyperbolic ones? Where do we begin?

\textbf{Acharya:} A natural starting point is the \textbf{wave
equation}:\\
\[
u_{tt} = c^2 u_{xx}.
\]

\textbf{Pavni:} Oh! That's the equation for vibrations of a string,
right?

\textbf{Acharya:} Exactly. Imagine a taut string. Newton's law applied
to a small element gives:\\
\[
\rho u_{tt} = T u_{xx},
\]\\
where \(T\) is tension and \(\rho\) is density. Dividing through, we
get\\
\[
u_{tt} = \left(\tfrac{T}{\rho}\right) u_{xx},
\]\\
so the wave speed is \(c = \sqrt{T/\rho}\).

\textbf{Pavni:} So physically, it describes oscillations moving along
the string. But why do we call this \emph{hyperbolic}?

\textbf{Acharya:} Let's look at the general second-order PDE in two
variables:\\
\[
A u_{xx} + 2B u_{xt} + C u_{tt} = 0.
\]\\
Its type depends on the discriminant \(D = B^2 - AC\).\\
- If \(D < 0\), it's elliptic.\\
- If \(D = 0\), parabolic.\\
- If \(D > 0\), hyperbolic.

\textbf{Pavni:} For the wave equation, we have \(A = -c^2, B=0, C=1\).
Then\\
\[
D = B^2 - AC = c^2 > 0.
\]

\textbf{Acharya:} Exactly --- that's why it's \textbf{hyperbolic}.

\textbf{Pavni:} Does that mean it has some special geometry?

\textbf{Acharya:} Yes. Notice how the operator factors:\\
\[
u_{tt} - c^2 u_{xx} = (\partial_t - c\partial_x)(\partial_t + c\partial_x) u.
\]\\
This reveals the \textbf{characteristics}: the lines\\
\[
x - ct = \text{constant}, \quad x + ct = \text{constant}.
\]

\textbf{Pavni:} So along those lines, the solution behaves in a simple
way?

\textbf{Acharya:} Precisely. The general solution is\\
\[
u(x,t) = f(x - ct) + g(x + ct).
\]\\
It's just the sum of two waves, one traveling right, one traveling left.

\textbf{Pavni:} That's beautiful! It really captures the idea of
finite-speed propagation.

\textbf{Acharya:} Indeed. If you disturb the string at one point, the
influence spreads only within the cone \(|x-x_0| \leq c(t-t_0)\).\\
That's the hallmark of hyperbolic PDEs: signals travel with finite
speed, unlike diffusion where influence is instant.

\textbf{Pavni:} So the wave equation is the prototype for hyperbolic
PDEs?

\textbf{Acharya:} Exactly. From here, we can explore more complex
hyperbolic equations --- nonlinear ones, conservation laws, and shock
waves --- but the wave equation is where the story begins.

\begin{Shaded}
\begin{Highlighting}[]
\NormalTok{\textbackslash{}begin\{tikzpicture\}[scale=1.2]}
\NormalTok{  \% Axes}
\NormalTok{  \textbackslash{}draw[{-}\textgreater{}] ({-}2,0) {-}{-} (2,0) node[right] \{$x$\};}
\NormalTok{  \textbackslash{}draw[{-}\textgreater{}] (0,0) {-}{-} (0,3) node[above] \{$t$\};}

\NormalTok{  \% Point of disturbance}
\NormalTok{  \textbackslash{}fill (0,0) circle (2pt) node[below left] \{$(x\_0,t\_0)$\};}

\NormalTok{  \% Characteristics}
\NormalTok{  \textbackslash{}draw[thick,red] (0,0) {-}{-} (1.5,3) node[right] \{$x = x\_0 + c(t{-}t\_0)$\};}
\NormalTok{  \textbackslash{}draw[thick,red] (0,0) {-}{-} ({-}1.5,3) node[left] \{$x = x\_0 {-} c(t{-}t\_0)$\};}

\NormalTok{  \% Shaded influence region}
\NormalTok{  \textbackslash{}fill[blue!20,opacity=0.5] ({-}1.5,3) {-}{-} (0,0) {-}{-} (1.5,3) {-}{-} cycle;}

\NormalTok{  \% Labels}
\NormalTok{  \textbackslash{}node at (0,1.8) \{Region of influence\};}
\NormalTok{\textbackslash{}end\{tikzpicture\}}
\end{Highlighting}
\end{Shaded}



\backmatter


\end{document}
