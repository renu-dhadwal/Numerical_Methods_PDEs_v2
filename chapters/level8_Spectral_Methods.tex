% Options for packages loaded elsewhere
% Options for packages loaded elsewhere
\PassOptionsToPackage{unicode}{hyperref}
\PassOptionsToPackage{hyphens}{url}
%
\documentclass[
  english,
]{book}
\usepackage{xcolor}
\usepackage{amsmath,amssymb}
\setcounter{secnumdepth}{-\maxdimen} % remove section numbering
\usepackage{iftex}
\ifPDFTeX
  \usepackage[T1]{fontenc}
  \usepackage[utf8]{inputenc}
  \usepackage{textcomp} % provide euro and other symbols
\else % if luatex or xetex
  \usepackage{unicode-math} % this also loads fontspec
  \defaultfontfeatures{Scale=MatchLowercase}
  \defaultfontfeatures[\rmfamily]{Ligatures=TeX,Scale=1}
\fi
\usepackage{lmodern}
\ifPDFTeX\else
  % xetex/luatex font selection
\fi
% Use upquote if available, for straight quotes in verbatim environments
\IfFileExists{upquote.sty}{\usepackage{upquote}}{}
\IfFileExists{microtype.sty}{% use microtype if available
  \usepackage[]{microtype}
  \UseMicrotypeSet[protrusion]{basicmath} % disable protrusion for tt fonts
}{}
\makeatletter
\@ifundefined{KOMAClassName}{% if non-KOMA class
  \IfFileExists{parskip.sty}{%
    \usepackage{parskip}
  }{% else
    \setlength{\parindent}{0pt}
    \setlength{\parskip}{6pt plus 2pt minus 1pt}}
}{% if KOMA class
  \KOMAoptions{parskip=half}}
\makeatother
% Make \paragraph and \subparagraph free-standing
\makeatletter
\ifx\paragraph\undefined\else
  \let\oldparagraph\paragraph
  \renewcommand{\paragraph}{
    \@ifstar
      \xxxParagraphStar
      \xxxParagraphNoStar
  }
  \newcommand{\xxxParagraphStar}[1]{\oldparagraph*{#1}\mbox{}}
  \newcommand{\xxxParagraphNoStar}[1]{\oldparagraph{#1}\mbox{}}
\fi
\ifx\subparagraph\undefined\else
  \let\oldsubparagraph\subparagraph
  \renewcommand{\subparagraph}{
    \@ifstar
      \xxxSubParagraphStar
      \xxxSubParagraphNoStar
  }
  \newcommand{\xxxSubParagraphStar}[1]{\oldsubparagraph*{#1}\mbox{}}
  \newcommand{\xxxSubParagraphNoStar}[1]{\oldsubparagraph{#1}\mbox{}}
\fi
\makeatother

\usepackage{color}
\usepackage{fancyvrb}
\newcommand{\VerbBar}{|}
\newcommand{\VERB}{\Verb[commandchars=\\\{\}]}
\DefineVerbatimEnvironment{Highlighting}{Verbatim}{commandchars=\\\{\}}
% Add ',fontsize=\small' for more characters per line
\usepackage{framed}
\definecolor{shadecolor}{RGB}{241,243,245}
\newenvironment{Shaded}{\begin{snugshade}}{\end{snugshade}}
\newcommand{\AlertTok}[1]{\textcolor[rgb]{0.68,0.00,0.00}{#1}}
\newcommand{\AnnotationTok}[1]{\textcolor[rgb]{0.37,0.37,0.37}{#1}}
\newcommand{\AttributeTok}[1]{\textcolor[rgb]{0.40,0.45,0.13}{#1}}
\newcommand{\BaseNTok}[1]{\textcolor[rgb]{0.68,0.00,0.00}{#1}}
\newcommand{\BuiltInTok}[1]{\textcolor[rgb]{0.00,0.23,0.31}{#1}}
\newcommand{\CharTok}[1]{\textcolor[rgb]{0.13,0.47,0.30}{#1}}
\newcommand{\CommentTok}[1]{\textcolor[rgb]{0.37,0.37,0.37}{#1}}
\newcommand{\CommentVarTok}[1]{\textcolor[rgb]{0.37,0.37,0.37}{\textit{#1}}}
\newcommand{\ConstantTok}[1]{\textcolor[rgb]{0.56,0.35,0.01}{#1}}
\newcommand{\ControlFlowTok}[1]{\textcolor[rgb]{0.00,0.23,0.31}{\textbf{#1}}}
\newcommand{\DataTypeTok}[1]{\textcolor[rgb]{0.68,0.00,0.00}{#1}}
\newcommand{\DecValTok}[1]{\textcolor[rgb]{0.68,0.00,0.00}{#1}}
\newcommand{\DocumentationTok}[1]{\textcolor[rgb]{0.37,0.37,0.37}{\textit{#1}}}
\newcommand{\ErrorTok}[1]{\textcolor[rgb]{0.68,0.00,0.00}{#1}}
\newcommand{\ExtensionTok}[1]{\textcolor[rgb]{0.00,0.23,0.31}{#1}}
\newcommand{\FloatTok}[1]{\textcolor[rgb]{0.68,0.00,0.00}{#1}}
\newcommand{\FunctionTok}[1]{\textcolor[rgb]{0.28,0.35,0.67}{#1}}
\newcommand{\ImportTok}[1]{\textcolor[rgb]{0.00,0.46,0.62}{#1}}
\newcommand{\InformationTok}[1]{\textcolor[rgb]{0.37,0.37,0.37}{#1}}
\newcommand{\KeywordTok}[1]{\textcolor[rgb]{0.00,0.23,0.31}{\textbf{#1}}}
\newcommand{\NormalTok}[1]{\textcolor[rgb]{0.00,0.23,0.31}{#1}}
\newcommand{\OperatorTok}[1]{\textcolor[rgb]{0.37,0.37,0.37}{#1}}
\newcommand{\OtherTok}[1]{\textcolor[rgb]{0.00,0.23,0.31}{#1}}
\newcommand{\PreprocessorTok}[1]{\textcolor[rgb]{0.68,0.00,0.00}{#1}}
\newcommand{\RegionMarkerTok}[1]{\textcolor[rgb]{0.00,0.23,0.31}{#1}}
\newcommand{\SpecialCharTok}[1]{\textcolor[rgb]{0.37,0.37,0.37}{#1}}
\newcommand{\SpecialStringTok}[1]{\textcolor[rgb]{0.13,0.47,0.30}{#1}}
\newcommand{\StringTok}[1]{\textcolor[rgb]{0.13,0.47,0.30}{#1}}
\newcommand{\VariableTok}[1]{\textcolor[rgb]{0.07,0.07,0.07}{#1}}
\newcommand{\VerbatimStringTok}[1]{\textcolor[rgb]{0.13,0.47,0.30}{#1}}
\newcommand{\WarningTok}[1]{\textcolor[rgb]{0.37,0.37,0.37}{\textit{#1}}}

\usepackage{longtable,booktabs,array}
\usepackage{calc} % for calculating minipage widths
% Correct order of tables after \paragraph or \subparagraph
\usepackage{etoolbox}
\makeatletter
\patchcmd\longtable{\par}{\if@noskipsec\mbox{}\fi\par}{}{}
\makeatother
% Allow footnotes in longtable head/foot
\IfFileExists{footnotehyper.sty}{\usepackage{footnotehyper}}{\usepackage{footnote}}
\makesavenoteenv{longtable}
\usepackage{graphicx}
\makeatletter
\newsavebox\pandoc@box
\newcommand*\pandocbounded[1]{% scales image to fit in text height/width
  \sbox\pandoc@box{#1}%
  \Gscale@div\@tempa{\textheight}{\dimexpr\ht\pandoc@box+\dp\pandoc@box\relax}%
  \Gscale@div\@tempb{\linewidth}{\wd\pandoc@box}%
  \ifdim\@tempb\p@<\@tempa\p@\let\@tempa\@tempb\fi% select the smaller of both
  \ifdim\@tempa\p@<\p@\scalebox{\@tempa}{\usebox\pandoc@box}%
  \else\usebox{\pandoc@box}%
  \fi%
}
% Set default figure placement to htbp
\def\fps@figure{htbp}
\makeatother



\ifLuaTeX
\usepackage[bidi=basic]{babel}
\else
\usepackage[bidi=default]{babel}
\fi
% get rid of language-specific shorthands (see #6817):
\let\LanguageShortHands\languageshorthands
\def\languageshorthands#1{}
\ifLuaTeX
  \usepackage[english]{selnolig} % disable illegal ligatures
\fi


\setlength{\emergencystretch}{3em} % prevent overfull lines

\providecommand{\tightlist}{%
  \setlength{\itemsep}{0pt}\setlength{\parskip}{0pt}}



 


% ---- Basic math setup ----
\usepackage{fontspec} % keep for Unicode, harmless even without custom fonts
% \setmainfont{TeX Gyre Pagella}      % comment out
% \setmathfont{TeX Gyre Pagella Math} % comment out

% ---- Disable emojis ----
\newcommand{\emoji}[1]{}

% ---- Math packages ----
\usepackage{amsmath,amssymb}
\usepackage{tikz}
\usetikzlibrary{arrows.meta,shapes,positioning}
\makeatletter
\@ifpackageloaded{tcolorbox}{}{\usepackage[skins,breakable]{tcolorbox}}
\@ifpackageloaded{fontawesome5}{}{\usepackage{fontawesome5}}
\definecolor{quarto-callout-color}{HTML}{909090}
\definecolor{quarto-callout-note-color}{HTML}{0758E5}
\definecolor{quarto-callout-important-color}{HTML}{CC1914}
\definecolor{quarto-callout-warning-color}{HTML}{EB9113}
\definecolor{quarto-callout-tip-color}{HTML}{00A047}
\definecolor{quarto-callout-caution-color}{HTML}{FC5300}
\definecolor{quarto-callout-color-frame}{HTML}{acacac}
\definecolor{quarto-callout-note-color-frame}{HTML}{4582ec}
\definecolor{quarto-callout-important-color-frame}{HTML}{d9534f}
\definecolor{quarto-callout-warning-color-frame}{HTML}{f0ad4e}
\definecolor{quarto-callout-tip-color-frame}{HTML}{02b875}
\definecolor{quarto-callout-caution-color-frame}{HTML}{fd7e14}
\makeatother
\makeatletter
\@ifpackageloaded{caption}{}{\usepackage{caption}}
\AtBeginDocument{%
\ifdefined\contentsname
  \renewcommand*\contentsname{Table of contents}
\else
  \newcommand\contentsname{Table of contents}
\fi
\ifdefined\listfigurename
  \renewcommand*\listfigurename{List of Figures}
\else
  \newcommand\listfigurename{List of Figures}
\fi
\ifdefined\listtablename
  \renewcommand*\listtablename{List of Tables}
\else
  \newcommand\listtablename{List of Tables}
\fi
\ifdefined\figurename
  \renewcommand*\figurename{Figure}
\else
  \newcommand\figurename{Figure}
\fi
\ifdefined\tablename
  \renewcommand*\tablename{Table}
\else
  \newcommand\tablename{Table}
\fi
}
\@ifpackageloaded{float}{}{\usepackage{float}}
\floatstyle{ruled}
\@ifundefined{c@chapter}{\newfloat{codelisting}{h}{lop}}{\newfloat{codelisting}{h}{lop}[chapter]}
\floatname{codelisting}{Listing}
\newcommand*\listoflistings{\listof{codelisting}{List of Listings}}
\makeatother
\makeatletter
\makeatother
\makeatletter
\@ifpackageloaded{caption}{}{\usepackage{caption}}
\@ifpackageloaded{subcaption}{}{\usepackage{subcaption}}
\makeatother
\usepackage{bookmark}
\IfFileExists{xurl.sty}{\usepackage{xurl}}{} % add URL line breaks if available
\urlstyle{same}
\hypersetup{
  pdftitle={From Finite Differences to Spectral Accuracy},
  pdflang={en},
  hidelinks,
  pdfcreator={LaTeX via pandoc}}


\title{From Finite Differences to Spectral Accuracy}
\author{}
\date{}
\begin{document}
\frontmatter
\maketitle


\mainmatter
\textbf{Pavni:} Acharya, last week you mentioned \emph{spectral
methods}. I've heard they're very accurate for smooth problems.
But\ldots{} how are they different from finite differences or finite
elements?

\textbf{Acharya:} A good question, Pavni. Let's start from what you
already know. In \textbf{finite difference} methods, we approximate
derivatives \emph{locally} using Taylor expansions --- yes?

\textbf{Pavni:} Yes. We replace \(u'(x)\) by something like
\(\tfrac{u(x+h) - u(x-h)}{2h}\), and the error goes as \(O(h^2)\) if the
function is smooth.

\textbf{Acharya:} Excellent. So, each approximation uses only
\emph{neighboring points}. Finite elements do something similar ---
piecewise approximations over small elements. But tell me --- if your
function is perfectly smooth, say \(u(x) = e^x\), do you really need to
treat each small interval separately?

\textbf{Pavni:} Hmm\ldots{} maybe not. If the function is smooth
everywhere, I could use one large polynomial to approximate it globally.

\textbf{Acharya:} Precisely. That is the spirit of \emph{spectral
methods.} Instead of stitching local approximations, we represent the
whole function as a sum of \emph{global basis functions}: \[
u_N(x) = \sum_{k=0}^{N} a_k \, \phi_k(x).
\]

\textbf{Pavni:} So the \(\phi_k\)'s are like the building blocks?

\textbf{Acharya:} Exactly. For periodic problems, we might take
\(\phi_k(x) = e^{ikx}\) --- the Fourier basis. For problems on
\([-1,1]\), we prefer \emph{orthogonal polynomials} such as Chebyshev or
Legendre polynomials.

\textbf{Pavni:} Then \(a_k\) are like coefficients of a series
expansion?

\textbf{Acharya:} Yes --- determined by enforcing the PDE or by matching
the function values at specific points, called \emph{collocation
points.}

\textbf{Pavni:} But Acharya, won't a global polynomial oscillate a lot,
like in the Runge phenomenon?

\textbf{Acharya:} Ah, you've been observant! The Runge phenomenon indeed
haunts equispaced interpolation. Let's recall:
\(f(x) = \frac{1}{1 + 25x^2}\). Interpolate it with a degree-10
polynomial at equally spaced points --- what happens?

\textbf{Pavni:} The polynomial wiggles wildly near the boundaries.

\textbf{Acharya:} Exactly. But there's a cure: use \textbf{Chebyshev
points}, clustered near \(-1\) and \(1\): \[
x_j = \cos\!\left(\frac{j\pi}{N}\right), \quad j = 0,1,\dots,N.
\] This choice minimizes the interpolation error and stabilizes the
approximation.

\textbf{Pavni:} So spacing the points unevenly makes the polynomial
behave better?

\textbf{Acharya:} Very much so. Think of it as giving more attention to
the troublesome boundary regions.

\textbf{Pavni:} You said spectral methods are \emph{very accurate.} How
accurate are we talking?

\textbf{Acharya:} If the function is smooth, the error decays
\textbf{exponentially} with the number of modes \(N\). Compare that with
finite differences, where the error decays algebraically --- like
\(O(N^{-2})\) or \(O(N^{-4})\). This exponential behavior is called
\textbf{spectral convergence}.

\textbf{Pavni:} That's amazing! So doubling the number of points can
reduce the error by orders of magnitude.

\textbf{Acharya:} Exactly. But remember --- only when the solution is
smooth. If the solution has discontinuities or sharp corners, spectral
methods lose their magic.

\textbf{Pavni:} I see. So the basis choice matters: Fourier for
periodic, Chebyshev for nonperiodic.

\textbf{Acharya:} Well summarized. Let's take an example to make this
concrete.

\subsection{Example: 1D Poisson
Equation}\label{example-1d-poisson-equation}

\[
u''(x) = \sin(\pi x), \qquad u(-1) = u(1) = 0.
\]

\textbf{Acharya:} We'll approximate \(u(x)\) at the
Chebyshev--Gauss--Lobatto points \(x_j\). Then we build the
\emph{differentiation matrix} \(D\), whose entries satisfy \[
u'(x_j) \approx \sum_{k=0}^{N} D_{jk} \, u(x_k),
\] and hence \(D^2 u \approx u''(x_j)\).

\textbf{Pavni:} So instead of computing derivatives analytically, we use
matrix multiplication?

\textbf{Acharya:} Exactly. The discrete derivative is a \emph{linear
transformation.} Once we have \(D^2\), we simply solve the linear system
\(D^2 u = f\), adjusting the first and last rows for boundary
conditions.

\textbf{Pavni:} That seems surprisingly compact.

\textbf{Acharya:} Indeed. The entire solver can be written in less than
10 lines of MATLAB or Python. That's the power of the \emph{spectral
collocation method.}

\textbf{Pavni:} How does it compare with finite differences?

\textbf{Acharya:} For the same \(N = 20\), finite difference might give
an error of \(10^{-2}\), while the spectral method can reach
\(10^{-10}\). Both use the same number of grid points!

\textbf{Pavni:} That's\ldots{} spectacular. Now I understand why they're
called \emph{spectral} methods.

\textbf{Acharya:} (smiles) A fitting pun, Pavni. The term actually comes
from \emph{spectrum} --- the set of basis functions, much like
frequencies in Fourier analysis.

\textbf{Pavni:} Are there any downsides?

\textbf{Acharya:} Yes. Spectral methods can struggle with non-smooth
problems, complex geometries, or localized features. But for smooth
flows --- such as those in polymer extrusion or viscoelastic jets ---
they shine brilliantly.

\textbf{Pavni:} That connects beautifully to your research!

\textbf{Acharya:} Exactly. In the next lecture, we'll derive the
\textbf{Chebyshev differentiation matrix} and use it to solve PDEs ---
perhaps even the heat equation.

\textbf{Pavni:} Wonderful. So today's takeaways are: 1. Spectral methods
use \emph{global} basis functions.\\
2. Chebyshev points stabilize interpolation.\\
3. For smooth problems, errors decay \emph{exponentially}.\\
4. Derivatives are represented by \emph{differentiation matrices}.

\textbf{Acharya:} Perfect summary, Pavni. Now, try plotting the
interpolation of \(e^x\) using Chebyshev points and observe the error.
You'll \emph{see} spectral convergence with your own eyes.

\section{References}\label{references}

\begin{itemize}
\tightlist
\item
  L. N. Trefethen, \emph{Spectral Methods in MATLAB} (SIAM).\\
\item
  J. P. Boyd, \emph{Chebyshev and Fourier Spectral Methods}.\\
\item
  Canuto et al., \emph{Spectral Methods: Fundamentals in Single
  Domains}.
\end{itemize}

\section{Demos (Optional)}\label{demos-optional}

\begin{tcolorbox}[enhanced jigsaw, colframe=quarto-callout-note-color-frame, colback=white, colbacktitle=quarto-callout-note-color!10!white, toptitle=1mm, arc=.35mm, opacitybacktitle=0.6, bottomtitle=1mm, titlerule=0mm, left=2mm, breakable, opacityback=0, title=\textcolor{quarto-callout-note-color}{\faInfo}\hspace{0.5em}{Note}, toprule=.15mm, leftrule=.75mm, bottomrule=.15mm, rightrule=.15mm, coltitle=black]

These sections are optional in-class or homework demos. They're
collapsed by default.

\end{tcolorbox}

Demo 1 --- Runge phenomenon vs.~Chebyshev nodes (Python)

\begin{Shaded}
\begin{Highlighting}[]
\CommentTok{\#| label: fig{-}runge}
\CommentTok{\#| fig{-}cap: "Runge phenomenon at equispaced points vs. stability at Chebyshev points."}
\ImportTok{import}\NormalTok{ numpy }\ImportTok{as}\NormalTok{ np, matplotlib.pyplot }\ImportTok{as}\NormalTok{ plt}

\NormalTok{f }\OperatorTok{=} \KeywordTok{lambda}\NormalTok{ x: }\DecValTok{1}\OperatorTok{/}\NormalTok{(}\DecValTok{1} \OperatorTok{+} \DecValTok{25}\OperatorTok{*}\NormalTok{x}\OperatorTok{**}\DecValTok{2}\NormalTok{)}
\NormalTok{xx }\OperatorTok{=}\NormalTok{ np.linspace(}\OperatorTok{{-}}\DecValTok{1}\NormalTok{, }\DecValTok{1}\NormalTok{, }\DecValTok{1000}\NormalTok{)}
\NormalTok{plt.figure()}
\NormalTok{plt.plot(xx, f(xx), lw}\OperatorTok{=}\DecValTok{2}\NormalTok{, label}\OperatorTok{=}\StringTok{"f(x)"}\NormalTok{)}

\ControlFlowTok{for}\NormalTok{ N }\KeywordTok{in}\NormalTok{ [}\DecValTok{5}\NormalTok{, }\DecValTok{10}\NormalTok{, }\DecValTok{15}\NormalTok{]:}
    \CommentTok{\# Equispaced}
\NormalTok{    x\_eq }\OperatorTok{=}\NormalTok{ np.linspace(}\OperatorTok{{-}}\DecValTok{1}\NormalTok{, }\DecValTok{1}\NormalTok{, N}\OperatorTok{+}\DecValTok{1}\NormalTok{)}
\NormalTok{    p\_eq }\OperatorTok{=}\NormalTok{ np.poly1d(np.polyfit(x\_eq, f(x\_eq), N))}
\NormalTok{    plt.plot(xx, p\_eq(xx), linestyle}\OperatorTok{=}\StringTok{"{-}{-}"}\NormalTok{, label}\OperatorTok{=}\SpecialStringTok{f"Equispaced N=}\CharTok{\{\{}\SpecialStringTok{N}\CharTok{\}\}}\SpecialStringTok{"}\NormalTok{)}

\ControlFlowTok{for}\NormalTok{ N }\KeywordTok{in}\NormalTok{ [}\DecValTok{5}\NormalTok{, }\DecValTok{10}\NormalTok{, }\DecValTok{15}\NormalTok{]:}
    \CommentTok{\# Chebyshev}
\NormalTok{    j }\OperatorTok{=}\NormalTok{ np.arange(N}\OperatorTok{+}\DecValTok{1}\NormalTok{)}
\NormalTok{    x\_ch }\OperatorTok{=}\NormalTok{ np.cos(np.pi }\OperatorTok{*}\NormalTok{ j }\OperatorTok{/}\NormalTok{ N)}
\NormalTok{    p\_ch }\OperatorTok{=}\NormalTok{ np.poly1d(np.polyfit(x\_ch, f(x\_ch), N))}
\NormalTok{    plt.plot(xx, p\_ch(xx), label}\OperatorTok{=}\SpecialStringTok{f"Chebyshev N=}\CharTok{\{\{}\SpecialStringTok{N}\CharTok{\}\}}\SpecialStringTok{"}\NormalTok{)}
\NormalTok{plt.legend()}\OperatorTok{;}\NormalTok{ plt.title(}\StringTok{"Runge vs. Chebyshev"}\NormalTok{)}\OperatorTok{;}\NormalTok{ plt.xlabel(}\StringTok{"x"}\NormalTok{)}\OperatorTok{;}\NormalTok{ plt.ylabel(}\StringTok{"y"}\NormalTok{)}
\NormalTok{plt.show()}
\end{Highlighting}
\end{Shaded}

Demo 2 --- Exponential convergence for \(e^x\) (Python)

\begin{Shaded}
\begin{Highlighting}[]
\CommentTok{\#| label: fig{-}exp{-}conv}
\CommentTok{\#| fig{-}cap: "Interpolation error for e\^{}x at Chebyshev nodes shows near{-}exponential decay."}
\ImportTok{import}\NormalTok{ numpy }\ImportTok{as}\NormalTok{ np, matplotlib.pyplot }\ImportTok{as}\NormalTok{ plt}

\NormalTok{f }\OperatorTok{=} \KeywordTok{lambda}\NormalTok{ x: np.exp(x)}
\NormalTok{xx }\OperatorTok{=}\NormalTok{ np.linspace(}\OperatorTok{{-}}\DecValTok{1}\NormalTok{,}\DecValTok{1}\NormalTok{,}\DecValTok{2000}\NormalTok{)}
\NormalTok{fxx }\OperatorTok{=}\NormalTok{ f(xx)}

\NormalTok{Ns }\OperatorTok{=} \BuiltInTok{range}\NormalTok{(}\DecValTok{4}\NormalTok{, }\DecValTok{40}\NormalTok{)}
\NormalTok{errs }\OperatorTok{=}\NormalTok{ []}
\ControlFlowTok{for}\NormalTok{ N }\KeywordTok{in}\NormalTok{ Ns:}
\NormalTok{    j }\OperatorTok{=}\NormalTok{ np.arange(N}\OperatorTok{+}\DecValTok{1}\NormalTok{)}
\NormalTok{    x }\OperatorTok{=}\NormalTok{ np.cos(np.pi }\OperatorTok{*}\NormalTok{ j }\OperatorTok{/}\NormalTok{ N)}
\NormalTok{    p }\OperatorTok{=}\NormalTok{ np.poly1d(np.polyfit(x, f(x), N))}
\NormalTok{    errs.append(np.}\BuiltInTok{max}\NormalTok{(np.}\BuiltInTok{abs}\NormalTok{(p(xx) }\OperatorTok{{-}}\NormalTok{ fxx)))}

\NormalTok{plt.figure()}
\NormalTok{plt.semilogy(}\BuiltInTok{list}\NormalTok{(Ns), errs, marker}\OperatorTok{=}\StringTok{\textquotesingle{}o\textquotesingle{}}\NormalTok{)}
\NormalTok{plt.xlabel(}\StringTok{"N"}\NormalTok{)}\OperatorTok{;}\NormalTok{ plt.ylabel(}\StringTok{"max error"}\NormalTok{)}\OperatorTok{;}\NormalTok{ plt.title(}\StringTok{"Spectral{-}like convergence for e\^{}x"}\NormalTok{)}
\NormalTok{plt.grid(}\VariableTok{True}\NormalTok{, which}\OperatorTok{=}\StringTok{"both"}\NormalTok{)}\OperatorTok{;}\NormalTok{ plt.show()}
\end{Highlighting}
\end{Shaded}

Demo 3 --- Chebyshev-collocation Poisson solver (MATLAB)

\begin{Shaded}
\begin{Highlighting}[]
\CommentTok{\% Chebyshev differentiation matrix (Trefethen\textquotesingle{}s cheb.m)}
\KeywordTok{function}\NormalTok{ [}\VariableTok{D}\OperatorTok{,}\VariableTok{x}\NormalTok{] }\OperatorTok{=} \VariableTok{cheb}\NormalTok{(}\VariableTok{N}\NormalTok{)}
    \KeywordTok{if} \VariableTok{N}\OperatorTok{==}\FloatTok{0}\OperatorTok{,} \VariableTok{D}\OperatorTok{=}\FloatTok{0}\OperatorTok{;} \VariableTok{x}\OperatorTok{=}\FloatTok{1}\OperatorTok{;} \KeywordTok{return}\OperatorTok{,} \KeywordTok{end}
    \VariableTok{x} \OperatorTok{=} \VariableTok{cos}\NormalTok{(}\VariableTok{pi}\OperatorTok{*}\NormalTok{(}\FloatTok{0}\OperatorTok{:}\VariableTok{N}\NormalTok{)}\OperatorTok{/}\VariableTok{N}\NormalTok{)}\OperatorTok{\textquotesingle{};} \VariableTok{c} \OperatorTok{=}\NormalTok{ [}\FloatTok{2}\OperatorTok{;} \VariableTok{ones}\NormalTok{(}\VariableTok{N}\OperatorTok{{-}}\FloatTok{1}\OperatorTok{,}\FloatTok{1}\NormalTok{)}\OperatorTok{;} \FloatTok{2}\NormalTok{] }\OperatorTok{.*}\NormalTok{ (}\OperatorTok{{-}}\FloatTok{1}\NormalTok{)}\OperatorTok{.\^{}}\NormalTok{(}\FloatTok{0}\OperatorTok{:}\VariableTok{N}\NormalTok{)}\OperatorTok{\textquotesingle{};}
    \VariableTok{X} \OperatorTok{=} \VariableTok{repmat}\NormalTok{(}\VariableTok{x}\OperatorTok{,}\FloatTok{1}\OperatorTok{,}\VariableTok{N}\OperatorTok{+}\FloatTok{1}\NormalTok{)}\OperatorTok{;} \VariableTok{dX} \OperatorTok{=} \VariableTok{X} \OperatorTok{{-}} \VariableTok{X}\OperatorTok{\textquotesingle{};}
    \VariableTok{D} \OperatorTok{=}\NormalTok{ (}\VariableTok{c}\OperatorTok{*}\NormalTok{(}\FloatTok{1}\OperatorTok{./}\VariableTok{c}\NormalTok{)}\OperatorTok{\textquotesingle{}}\NormalTok{)}\OperatorTok{./}\NormalTok{(}\VariableTok{dX} \OperatorTok{+}\NormalTok{ (}\VariableTok{eye}\NormalTok{(}\VariableTok{N}\OperatorTok{+}\FloatTok{1}\NormalTok{)))}\OperatorTok{;} \CommentTok{\% off{-}diagonal}
    \VariableTok{D} \OperatorTok{=} \VariableTok{D} \OperatorTok{{-}} \VariableTok{diag}\NormalTok{(}\VariableTok{sum}\NormalTok{(}\VariableTok{D}\OperatorTok{,}\FloatTok{2}\NormalTok{))}\OperatorTok{;}             \CommentTok{\% diagonal}
\KeywordTok{end}

\CommentTok{\% Poisson problem u\textquotesingle{}\textquotesingle{} = f with u({-}1)=u(1)=0}
\VariableTok{N} \OperatorTok{=} \FloatTok{20}\OperatorTok{;}\NormalTok{ [}\VariableTok{D}\OperatorTok{,}\VariableTok{x}\NormalTok{] }\OperatorTok{=} \VariableTok{cheb}\NormalTok{(}\VariableTok{N}\NormalTok{)}\OperatorTok{;} \VariableTok{D2} \OperatorTok{=} \VariableTok{D}\OperatorTok{\^{}}\FloatTok{2}\OperatorTok{;}
\VariableTok{f} \OperatorTok{=} \VariableTok{sin}\NormalTok{(}\VariableTok{pi}\OperatorTok{*}\VariableTok{x}\NormalTok{)}\OperatorTok{;}
\CommentTok{\% Enforce Dirichlet BCs}
\VariableTok{D2}\NormalTok{(}\FloatTok{1}\OperatorTok{,:}\NormalTok{) }\OperatorTok{=} \FloatTok{0}\OperatorTok{;} \VariableTok{D2}\NormalTok{(}\FloatTok{1}\OperatorTok{,}\FloatTok{1}\NormalTok{) }\OperatorTok{=} \FloatTok{1}\OperatorTok{;} \VariableTok{f}\NormalTok{(}\FloatTok{1}\NormalTok{) }\OperatorTok{=} \FloatTok{0}\OperatorTok{;}
\VariableTok{D2}\NormalTok{(}\KeywordTok{end}\OperatorTok{,:}\NormalTok{) }\OperatorTok{=} \FloatTok{0}\OperatorTok{;} \VariableTok{D2}\NormalTok{(}\KeywordTok{end}\OperatorTok{,}\KeywordTok{end}\NormalTok{) }\OperatorTok{=} \FloatTok{1}\OperatorTok{;} \VariableTok{f}\NormalTok{(}\KeywordTok{end}\NormalTok{) }\OperatorTok{=} \FloatTok{0}\OperatorTok{;}
\VariableTok{u} \OperatorTok{=} \VariableTok{D2}\OperatorTok{\textbackslash{}}\VariableTok{f}\OperatorTok{;}
\VariableTok{plot}\NormalTok{(}\VariableTok{x}\OperatorTok{,}\VariableTok{u}\OperatorTok{,}\SpecialStringTok{\textquotesingle{}{-}o\textquotesingle{}}\NormalTok{)}\OperatorTok{;} \VariableTok{xlabel}\NormalTok{(}\SpecialStringTok{\textquotesingle{}x\textquotesingle{}}\NormalTok{)}\OperatorTok{;} \VariableTok{ylabel}\NormalTok{(}\SpecialStringTok{\textquotesingle{}u\textquotesingle{}}\NormalTok{)}\OperatorTok{;} \VariableTok{title}\NormalTok{(}\SpecialStringTok{\textquotesingle{}Chebyshev collocation solution\textquotesingle{}}\NormalTok{)}
\end{Highlighting}
\end{Shaded}



\backmatter


\end{document}
