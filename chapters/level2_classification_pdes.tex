% Options for packages loaded elsewhere
% Options for packages loaded elsewhere
\PassOptionsToPackage{unicode}{hyperref}
\PassOptionsToPackage{hyphens}{url}
%
\documentclass[
]{book}
\usepackage{xcolor}
\usepackage{amsmath,amssymb}
\setcounter{secnumdepth}{-\maxdimen} % remove section numbering
\usepackage{iftex}
\ifPDFTeX
  \usepackage[T1]{fontenc}
  \usepackage[utf8]{inputenc}
  \usepackage{textcomp} % provide euro and other symbols
\else % if luatex or xetex
  \usepackage{unicode-math} % this also loads fontspec
  \defaultfontfeatures{Scale=MatchLowercase}
  \defaultfontfeatures[\rmfamily]{Ligatures=TeX,Scale=1}
\fi
\usepackage{lmodern}
\ifPDFTeX\else
  % xetex/luatex font selection
\fi
% Use upquote if available, for straight quotes in verbatim environments
\IfFileExists{upquote.sty}{\usepackage{upquote}}{}
\IfFileExists{microtype.sty}{% use microtype if available
  \usepackage[]{microtype}
  \UseMicrotypeSet[protrusion]{basicmath} % disable protrusion for tt fonts
}{}
\makeatletter
\@ifundefined{KOMAClassName}{% if non-KOMA class
  \IfFileExists{parskip.sty}{%
    \usepackage{parskip}
  }{% else
    \setlength{\parindent}{0pt}
    \setlength{\parskip}{6pt plus 2pt minus 1pt}}
}{% if KOMA class
  \KOMAoptions{parskip=half}}
\makeatother
% Make \paragraph and \subparagraph free-standing
\makeatletter
\ifx\paragraph\undefined\else
  \let\oldparagraph\paragraph
  \renewcommand{\paragraph}{
    \@ifstar
      \xxxParagraphStar
      \xxxParagraphNoStar
  }
  \newcommand{\xxxParagraphStar}[1]{\oldparagraph*{#1}\mbox{}}
  \newcommand{\xxxParagraphNoStar}[1]{\oldparagraph{#1}\mbox{}}
\fi
\ifx\subparagraph\undefined\else
  \let\oldsubparagraph\subparagraph
  \renewcommand{\subparagraph}{
    \@ifstar
      \xxxSubParagraphStar
      \xxxSubParagraphNoStar
  }
  \newcommand{\xxxSubParagraphStar}[1]{\oldsubparagraph*{#1}\mbox{}}
  \newcommand{\xxxSubParagraphNoStar}[1]{\oldsubparagraph{#1}\mbox{}}
\fi
\makeatother


\usepackage{longtable,booktabs,array}
\usepackage{calc} % for calculating minipage widths
% Correct order of tables after \paragraph or \subparagraph
\usepackage{etoolbox}
\makeatletter
\patchcmd\longtable{\par}{\if@noskipsec\mbox{}\fi\par}{}{}
\makeatother
% Allow footnotes in longtable head/foot
\IfFileExists{footnotehyper.sty}{\usepackage{footnotehyper}}{\usepackage{footnote}}
\makesavenoteenv{longtable}
\usepackage{graphicx}
\makeatletter
\newsavebox\pandoc@box
\newcommand*\pandocbounded[1]{% scales image to fit in text height/width
  \sbox\pandoc@box{#1}%
  \Gscale@div\@tempa{\textheight}{\dimexpr\ht\pandoc@box+\dp\pandoc@box\relax}%
  \Gscale@div\@tempb{\linewidth}{\wd\pandoc@box}%
  \ifdim\@tempb\p@<\@tempa\p@\let\@tempa\@tempb\fi% select the smaller of both
  \ifdim\@tempa\p@<\p@\scalebox{\@tempa}{\usebox\pandoc@box}%
  \else\usebox{\pandoc@box}%
  \fi%
}
% Set default figure placement to htbp
\def\fps@figure{htbp}
\makeatother





\setlength{\emergencystretch}{3em} % prevent overfull lines

\providecommand{\tightlist}{%
  \setlength{\itemsep}{0pt}\setlength{\parskip}{0pt}}



 


% ---- Basic math setup ----
\usepackage{fontspec} % keep for Unicode, harmless even without custom fonts
% \setmainfont{TeX Gyre Pagella}      % comment out
% \setmathfont{TeX Gyre Pagella Math} % comment out

% ---- Disable emojis ----
\newcommand{\emoji}[1]{}

% ---- Math packages ----
\usepackage{amsmath,amssymb}
\usepackage{tikz}
\usetikzlibrary{arrows.meta,shapes,positioning}
\makeatletter
\@ifpackageloaded{tcolorbox}{}{\usepackage[skins,breakable]{tcolorbox}}
\@ifpackageloaded{fontawesome5}{}{\usepackage{fontawesome5}}
\definecolor{quarto-callout-color}{HTML}{909090}
\definecolor{quarto-callout-note-color}{HTML}{0758E5}
\definecolor{quarto-callout-important-color}{HTML}{CC1914}
\definecolor{quarto-callout-warning-color}{HTML}{EB9113}
\definecolor{quarto-callout-tip-color}{HTML}{00A047}
\definecolor{quarto-callout-caution-color}{HTML}{FC5300}
\definecolor{quarto-callout-color-frame}{HTML}{acacac}
\definecolor{quarto-callout-note-color-frame}{HTML}{4582ec}
\definecolor{quarto-callout-important-color-frame}{HTML}{d9534f}
\definecolor{quarto-callout-warning-color-frame}{HTML}{f0ad4e}
\definecolor{quarto-callout-tip-color-frame}{HTML}{02b875}
\definecolor{quarto-callout-caution-color-frame}{HTML}{fd7e14}
\makeatother
\makeatletter
\@ifpackageloaded{caption}{}{\usepackage{caption}}
\AtBeginDocument{%
\ifdefined\contentsname
  \renewcommand*\contentsname{Table of contents}
\else
  \newcommand\contentsname{Table of contents}
\fi
\ifdefined\listfigurename
  \renewcommand*\listfigurename{List of Figures}
\else
  \newcommand\listfigurename{List of Figures}
\fi
\ifdefined\listtablename
  \renewcommand*\listtablename{List of Tables}
\else
  \newcommand\listtablename{List of Tables}
\fi
\ifdefined\figurename
  \renewcommand*\figurename{Figure}
\else
  \newcommand\figurename{Figure}
\fi
\ifdefined\tablename
  \renewcommand*\tablename{Table}
\else
  \newcommand\tablename{Table}
\fi
}
\@ifpackageloaded{float}{}{\usepackage{float}}
\floatstyle{ruled}
\@ifundefined{c@chapter}{\newfloat{codelisting}{h}{lop}}{\newfloat{codelisting}{h}{lop}[chapter]}
\floatname{codelisting}{Listing}
\newcommand*\listoflistings{\listof{codelisting}{List of Listings}}
\makeatother
\makeatletter
\makeatother
\makeatletter
\@ifpackageloaded{caption}{}{\usepackage{caption}}
\@ifpackageloaded{subcaption}{}{\usepackage{subcaption}}
\makeatother
\usepackage{bookmark}
\IfFileExists{xurl.sty}{\usepackage{xurl}}{} % add URL line breaks if available
\urlstyle{same}
\hypersetup{
  hidelinks,
  pdfcreator={LaTeX via pandoc}}


\author{}
\date{}
\begin{document}
\frontmatter


\mainmatter
\chapter{Level 2: Classification of
PDEs}\label{level-2-classification-of-pdes}

\textbf{Pavni:} Acharya, last time you told me how PDEs were born from
strings, heat, and flows. But there seem to be so many kinds of PDEs.
How do we organize them?

\textbf{Acharya:} A good question, Pavni. Mathematicians classify PDEs
in several ways, much like a botanist classifies plants. Let us begin
with the simplest.

\textbf{Pavni:} I am listening.

\textbf{Acharya:} First, by \textbf{order}. A PDE is first-order if the
highest derivative is first order, second-order if the highest is
second, and so on. For example, the transport equation
\(u_t + c u_x = 0\) is first-order, while the heat equation
\(u_t = \alpha u_{xx}\) is second-order.

\textbf{Pavni:} That part seems simple enough. What else?

\textbf{Acharya:} Next comes \textbf{linearity}. A PDE is linear if the
unknown function and its derivatives appear only linearly --- not
multiplied together, not inside a sine or square. The heat equation is
linear. But if you had a term like \(u u_x\), that would make it
nonlinear.

\textbf{Pavni:} So, nonlinear PDEs are trickier?

\textbf{Acharya:} Very much so! Nonlinearity makes life both harder and
richer.

\textbf{Pavni:} Are there other distinctions?

\textbf{Acharya:} Yes. Equations can be \textbf{homogeneous} or
\textbf{inhomogeneous} depending on whether the right-hand side is zero.
Laplace's equation is homogeneous, Poisson's equation is inhomogeneous.

\textbf{Acharya:} More generally, a non-homogeneous PDE can be written
as:\\
\[ F(t, x_1, x_2, \dots, x_n, u, u_t, u_{x_1}, \dots, u_{x_n}, u_{tt}, u_{x_1x_1}, \dots) = g(x_1, x_2, \dots, x_n, t), \]\\
where \(g\) is a nonzero source or forcing term. If \(g \equiv 0\), the
PDE is homogeneous.

\textbf{Pavni:} So the right-hand side introduces an external influence,
like heat sources or forces?

\textbf{Acharya:} Exactly. For example, \(u_{xx} + u_{yy} = f(x,y)\) is
the Poisson equation with a source term \(f(x,y)\). .

\textbf{Pavni:} I think I follow. But I have also heard words like
elliptic and hyperbolic. What do they mean?

\textbf{Acharya:} Ah, those arise for \textbf{second-order PDEs in two
variables}. Suppose we have:\\
\[ A u_{xx} + B u_{xy} + C u_{yy} + \dots = 0. \]\\
We look at the discriminant: \(B^2 - 4AC\).\\
- If it is less than 0, the PDE is \textbf{elliptic} (like Laplace's
equation).\\
- If it equals 0, the PDE is \textbf{parabolic} (like the heat
equation).\\
- If it is greater than 0, the PDE is \textbf{hyperbolic} (like the wave
equation).

\textbf{Pavni:} This reminds me of conic sections! Circles, parabolas,
and hyperbolas.

\textbf{Acharya:} Exactly. The analogy is deliberate --- both come from
the same quadratic form.

\textbf{Pavni:} And physically?

\textbf{Acharya:} Elliptic equations describe steady states --- like
equilibrium temperature distributions. Parabolic equations describe
diffusion, the smoothing of irregularities over time. Hyperbolic
equations describe wave-like motion, signals traveling with finite
speed.

\textbf{Pavni:} That helps me imagine them. So classification is not
just a game, but it tells us the nature of the solutions.

\textbf{Acharya:} Well said, Pavni. And it also guides how we design
numerical methods. An elliptic problem requires different strategies
than a hyperbolic one.

\textbf{Pavni:} Then I am eager to learn those strategies!

\textbf{Acharya:} Patience, Pavni. First we must prepare the ground.
Classification is the map; numerical methods are the journey.

\begin{center}\rule{0.5\linewidth}{0.5pt}\end{center}

\section{Mini Quizzes}\label{mini-quizzes}

\textbf{Quiz 1:} Identify the order\\
Which of the following is a second-order PDE?\\
1. \$ u\_t + cu\_x = 0 \$\\
2. \$ u\_t = \alpha u\_\{xx\} \$\\
3. \$ u u\_x = 0 \$

\begin{tcolorbox}[enhanced jigsaw, left=2mm, bottomrule=.15mm, opacityback=0, rightrule=.15mm, leftrule=.75mm, coltitle=black, arc=.35mm, title=\textcolor{quarto-callout-tip-color}{\faLightbulb}\hspace{0.5em}{Answer 1}, bottomtitle=1mm, toptitle=1mm, breakable, opacitybacktitle=0.6, colback=white, colframe=quarto-callout-tip-color-frame, toprule=.15mm, titlerule=0mm, colbacktitle=quarto-callout-tip-color!10!white]

Equation (2) \$ u\_t = \alpha u\_\{xx\} \$ is second-order because of
the \(u_{xx}\) term.

\end{tcolorbox}

\begin{center}\rule{0.5\linewidth}{0.5pt}\end{center}

\textbf{Quiz 2:} Linearity check\\
Which PDE is nonlinear?\\
1. \$ u\_t = \alpha u\_\{xx\} \$\\
2. \$ u\_t + u u\_x = 0 \$

\begin{tcolorbox}[enhanced jigsaw, left=2mm, bottomrule=.15mm, opacityback=0, rightrule=.15mm, leftrule=.75mm, coltitle=black, arc=.35mm, title=\textcolor{quarto-callout-tip-color}{\faLightbulb}\hspace{0.5em}{Answer 2}, bottomtitle=1mm, toptitle=1mm, breakable, opacitybacktitle=0.6, colback=white, colframe=quarto-callout-tip-color-frame, toprule=.15mm, titlerule=0mm, colbacktitle=quarto-callout-tip-color!10!white]

Equation (2) is nonlinear because of the product term \(u u_x\).

\end{tcolorbox}

\begin{center}\rule{0.5\linewidth}{0.5pt}\end{center}

\textbf{Quiz 3:} Homogeneous or inhomogeneous\\
Classify: \$ u\_\{xx\} + u\_\{yy\} = f(x,y) \$.

\begin{tcolorbox}[enhanced jigsaw, left=2mm, bottomrule=.15mm, opacityback=0, rightrule=.15mm, leftrule=.75mm, coltitle=black, arc=.35mm, title=\textcolor{quarto-callout-tip-color}{\faLightbulb}\hspace{0.5em}{Answer 3}, bottomtitle=1mm, toptitle=1mm, breakable, opacitybacktitle=0.6, colback=white, colframe=quarto-callout-tip-color-frame, toprule=.15mm, titlerule=0mm, colbacktitle=quarto-callout-tip-color!10!white]

It is \textbf{inhomogeneous}, since the right-hand side is not zero.

\end{tcolorbox}

\begin{center}\rule{0.5\linewidth}{0.5pt}\end{center}

\textbf{Quiz 4:} Type of second-order PDE\\
For \$ u\_\{xx\} + 2u\_\{xy\} + u\_\{yy\} = 0 \$, compute \$ B\^{}2 -
4AC \$. What type is it?

\begin{tcolorbox}[enhanced jigsaw, left=2mm, bottomrule=.15mm, opacityback=0, rightrule=.15mm, leftrule=.75mm, coltitle=black, arc=.35mm, title=\textcolor{quarto-callout-tip-color}{\faLightbulb}\hspace{0.5em}{Answer 4}, bottomtitle=1mm, toptitle=1mm, breakable, opacitybacktitle=0.6, colback=white, colframe=quarto-callout-tip-color-frame, toprule=.15mm, titlerule=0mm, colbacktitle=quarto-callout-tip-color!10!white]

Here, \(A = 1, B = 2, C = 1\).\\
\$ B\^{}2 - 4AC = 2\^{}2 - 4(1)(1) = 0 \$.\\
So it is \textbf{parabolic}.

\end{tcolorbox}

\begin{center}\rule{0.5\linewidth}{0.5pt}\end{center}

\textbf{Quiz 5:} Physical meaning\\
Match each equation with its physical interpretation:\\
- Heat equation\\
- Wave equation\\
- Laplace's equation

\begin{enumerate}
\def\labelenumi{(\alph{enumi})}
\tightlist
\item
  Steady state\\
\item
  Wave-like motion\\
\item
  Diffusion in time
\end{enumerate}

\begin{tcolorbox}[enhanced jigsaw, left=2mm, bottomrule=.15mm, opacityback=0, rightrule=.15mm, leftrule=.75mm, coltitle=black, arc=.35mm, title=\textcolor{quarto-callout-tip-color}{\faLightbulb}\hspace{0.5em}{Answer 5}, bottomtitle=1mm, toptitle=1mm, breakable, opacitybacktitle=0.6, colback=white, colframe=quarto-callout-tip-color-frame, toprule=.15mm, titlerule=0mm, colbacktitle=quarto-callout-tip-color!10!white]

\begin{itemize}
\tightlist
\item
  Heat equation → (c) Diffusion in time\\
\item
  Wave equation → (b) Wave-like motion\\
\item
  Laplace's equation → (a) Steady state\\
\end{itemize}

\end{tcolorbox}


\backmatter


\end{document}
